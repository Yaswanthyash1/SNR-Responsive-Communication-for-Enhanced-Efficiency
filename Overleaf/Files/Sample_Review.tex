\documentclass[a4paper,10pt]{article}
\usepackage[english]{babel}
\usepackage[utf8]{inputenc}
\usepackage{setspace}
\usepackage{soul}
\usepackage{multirow}
\usepackage{array}
\usepackage{multicol}
\usepackage[export]{adjustbox}[2011/08/13]
\usepackage{tabularx}
% \usepackage{cite}
\usepackage{xcolor}
\usepackage{soul}
\usepackage{longtable}
\usepackage[numbers, sort&compress]{natbib}
% \usepackage[margin=0.8in]{geometry}
\usepackage[a4paper, total={6in, 10in}]{geometry}
\usepackage{array}
\newcolumntype{L}[1]{>{\raggedright\let\newline\\\arraybackslash\hspace{0pt}}m{#1}}
\newcolumntype{C}[1]{>{\centering\let\newline\\\arraybackslash\hspace{0pt}}m{#1}}
\newcolumntype{R}[1]{>{\raggedleft\let\newline\\\arraybackslash\hspace{0pt}}m{#1}}
\renewcommand*\labelenumi{\theenumi)}

%\usepackage[backend=bibtex,style=verbose-trad2]{biblatex}
\usepackage[ruled,norelsize]{algorithm2e}
\def\BibTeX{{\rm B\kern-.05em{\sc i\kern-.025em b}\kern-.08em
    T\kern-.1667em\lower.7ex\hbox{E}\kern-.125emX}}
\makeatletter
\newcommand{\removelatexerror}{\let\@latex@error\@gobble}
\makeatother

\date{}

\begin{document}

\setstretch{1.5}

%-------- Covering Letter --------------
	 	 	
\noindent From,\\
Write names,\\
Department of Computer Science and Engineering,\\
NITK, Surathkal, Mangalore – 575025, India.\\

\vspace{0.25in}

\noindent To,\\
Anshul Verma, Ph.D.\\
Guest Editor\\
SN Computer Science\\

\noindent Dear Sir/Madam,

\begin{center}
\textbf{Subject:} \textbf{Revised Manuscript} for SN Computer Science - Reg.
\end{center}

\noindent We are submitting herewith the revised manuscript  \textbf{SNCS-D-23-01695}, titled: \textbf{``Heterogeneous Data Integration of Synergetic IoT Applications with Heterogeneous Format Data for Interoperability using IBM-ACE''}. 

\vspace{0.25in}

\noindent We appreciate the Reviewers’ valuable comments and constructive suggestions and have accordingly incorporated the same. We sincerely believe this revised manuscript will be able to address all the issues raised during the review process.

\vspace{0.25in}

\noindent Thanking you, \\

\noindent Yours faithfully \\ 

\noindent Write names \\

\vspace{0.25in}
 
\noindent \textbf{Encl:} Responses to Reviewer \#2 and Reviewer \#3 Comments

% ----------------- End -----------------------

\newpage

\noindent \textbf{Responses to  Reviewer \#2 Comments}

\vspace{0.25in}

\noindent \textbf{Manuscript Title:} Heterogeneous Data Integration of Synergetic IoT Applications with Heterogeneous Format Data for Interoperability using IBM-ACE \\

\noindent \textbf{Manuscript Id:} SNCS-D-23-01695 \\


% 1
\begin{enumerate}

{\color{blue} \item \textbf{Comment-a:} The authors have implemented SVM, KNN, DT, and RF classifiers in their proposed model. I would suggest the authors to conduct the experiment with clustering techniques like Hierarchical, K-means etc.  In addition to this, I would also suggest the authors to try with others classifiers such as Random Forest and  so on.} 

\textbf{Response:} We appreciate the review's suggestions, and accordingly, Section 5.1 (Data Recognition) in the revised manuscript (Page No. 23 - 28) has been updated as follows,

\begin{itemize}
    \item Section 5.1.1 (Data Recognition using Clustering) is been updated with  K-means and Hierarchical Agglomerative clustering (HAC) algorithms. The clustering algorithm is applied to group the features of five data formats. The outcome is analyzed with various clustering metrics, and the performance of the two clustering algorithms is recorded in Table 4 (Page no. 24). Further, an ``elbow method'' was exercised to evaluate the K-means clustering output, and the same is recorded in Fig. 5 (Page. no. 25). As elbow method cannot be used to evaluate HAC outcome, a scatter plot is adopted. Fig. 6 (Page. no. 25) demonstrates the outcome of the HAC algorithm. It is observed that the clustering process results show a low metrics score indicating failure of K-means and HAC clustering to capture the underlying class information and group the data format features efficiently. Further, the elbow method also failed to provide the correct elbow point indicating the number of true clusters. Additionally, the scatter plot for HAC resulted in overlapped cluster points indicating clustering is not a reliable technique to recognize the data formats from their content. 
    
    \item Section 5.1.2 (Data Recognition with Classification) is been updated with state-of-the-art ML classifier models such as Support Vector Machine (SVM), K-Nearest Neighbour (KNN), Decision Tree (DT), Random Forest(RF), Extended Random Forest (ERF), Gaussian Naive Bayes (GNB), and XG Boost (XGB). To determine the best machine learning (ML) model for the proposed ISHII, the classifiers mentioned earlier were used to classify data formats based on their content. The results were compared to select the effective model. The classification results are obtained in two-phase, initially for three data formats and subsequently for five data formats, as shown in Fig. 7a and 7b (Page no. 26), respectively. Further, the models are compared with other evaluation metrics, and the outcome is recorded in Table 5 (Page. no. 27). Finally, the RF classifier selection for the proposed ISHII's data recognition is supported by the confusion matrix in Fig. 8 (Page. no. 28) and Table 6 (Page. no. 27).
    % \end{itemize}
\end{itemize}

% 3
{\color{blue} \item \textbf{Comment-b:} The authors have conducted the experiments using only three data sets. I would suggest the authors to test the proposed model using more number of data sets and prepare a graphical analysis of their model over the existing works for different data sets that is used in the work.} 
 
\textbf{Response:} We appreciate the reviewer's observation and comments. Accordingly, the existing work is updated with experiments using two additional new data formats, JSON-LD\footnote{https://json-ld.org/} and Google's Protocol Buffer (Protobuf)\footnote{https://cloud.google.com/apis/design/proto3}. The details in the revised manuscript are updated as follows:
\begin{itemize}
      \item Section 4.2 (Dataset for ML training and testing) is updated (Page no. 13 - 15) with details of dataset updation from an existing dataset with JSON, CSV, and XML data formats features to the new dataset with additional JSON-LD and Protobuf features. 
      \item Additionally, Section 5.1.2 (Data Recognition with Classification) is updated with the results from data recognition experiments conducted on various ML classifier models using existing and updated datasets. The corresponding results are updated as model accuracy comparison plots in Fig. 7a and 7b (Page no. 26). Further, the ML models performance metrics are presented in Table 5 (Page no. 27). Finally, the Data recognition module is evaluated with updated patient monitoring and room ambiance dataset, and the outcome is recorded as Confusion matrix in Table 6 (Page no. 27).  
\end{itemize}

\textbf{Note:} As a response to \textbf{Comment-a}, we have provided experimental results for additional data formats.

% 4
{\color{blue} \item \textbf{Comment-c:}  I would suggest the authors to compare their model with more existing works.} 

\textbf{Response:}  We appreciate the review's suggestions. Accordingly, the proposed ISHII approach is compared with other existing work based on parameters such as deployment complexity, usage of proprietary data formats, data loss due to format conversion, and data representation in an interoperable format. Subsequently, Section 4.5 (Comparison of ISHII with existing interoperability
approaches) is updated with Table 3 in the revised manuscript (Page no. 22 and 23).

{\color{blue} \item \textbf{Comment-d:}The authors should incorporate more details about the existing works in the related work section.}

\textbf{Response:}  We appreciate the review's suggestions. Accordingly, the revised manuscript's Section 2 (Related work) is updated (Page No. 5 - 7) with additional approaches adopted by authors Gonzalez-Usach et al. [19], Jaleel et al. [20], Ahmed et al. [21], Modoni et al. [22], and Singh et al. [23] to address the interoperability issue. 

\end{enumerate}


\newpage


\noindent \textbf{Responses to  Reviewer \#3 Comments}

\vspace{0.25in}

\noindent \textbf{Manuscript Title:} ``Heterogeneous Data Integration of Synergetic IoT Applications with Heterogeneous Format Data for Interoperability using IBM-ACE'' \\

\noindent \textbf{Manuscript Id:} SNCS-D-23-01695 \\

\begin{enumerate}
% 1
{\color{blue} \item \textbf{Comment-1:} Results and Discussion section is very weak. Authors should improve this section by including more results in graphical and tabular forms, and should compare proposed approach performance with existing approaches.}\\

\textbf{Response:}  We appreciate the reviewer's observation and suggestions. Accordingly, the revised manuscript is updated (Page no. 23 - 30) with additional experiment results, presented in graphical and tabular format in Section 5 (Results and Discussion). Additionally, the proposed ISHII approach is compared with popular existing works from the literature and recorded in Section 4.5 (Comparison of ISHII with existing interoperability approaches) Table. 3 in the revised manuscript (Page No. 22 - 23). The details of the updations to the Section 5 (Result and Discussion) are as follows,

\begin{itemize}
    \item Section 5.1.1 (Data Recognition using Clustering) is been updated with the results of K-means and Hierarchical Agglomerative clustering (HAC) using an updated dataset with five format features. The performance of the two clustering algorithms is recorded in Table 4 (Page no. 24). Further, K-means clustering output is presented in elbow method  Fig. 5 (Page. no. 25). Additionally, Fig. 6 (Page. no. 25) demonstrating the outcome of the HAC algorithm.
    \item Section 5.1.2 (Data Recognition with Classification) is updated with classification results of state-of-the-art ML classifier models. The model accuracy comparison plots for an experiment on the existing dataset and new dataset results are updated as Fig. 7a and 7b (Page no. 26). Further, the classifier models are compared with other evaluation metrics and the outcome is recorded in Table 5 (Page. no. 27). Additionally, the RF classifier selection for proposed ISHII's data recognition is supported by confusion matrix in Fig. 8 (Page. no. 28) and Table 6 (Page. no. 27).
    \item Section 5.3 (Translation) is updated with sample output in new Protobuf format as Fig. 10c (Page no. 31). 
\end{itemize}


\end{enumerate}

\end{document}